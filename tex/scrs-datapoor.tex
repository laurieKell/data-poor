% $Id: $
\documentclass[a4paper, 10pt]{article}
% reduced margins
\usepackage{fullpage}
\usepackage[authoryear]{natbib}
% spacing
\usepackage{setspace}
% page headings
\usepackage{fancyhdr}
%\usepackage{lscape}


\usepackage{enumerate}	

\setlength{\headheight}{15.2pt}
\pagestyle{fancy}
% urls

\usepackage{lscape}
\usepackage{graphicx}
\usepackage{color}
\usepackage{hyperref}
\usepackage{url}
\hypersetup{colorlinks, urlcolor=darkblue}


\usepackage{lscape}
% figs to be 75% of test width
\setkeys{Gin}{width=0.75\textwidth}


%
\renewcommand{\abstractname}{\large SUMMARY}
%
\newcommand{\Keywords}[1]{\begin{center}\par\noindent{{\em KEYWORDS\/}: #1}\end{center}}
%
\makeatletter
\renewcommand{\subsubsection}{\@startsection{subsubsection}{3}{\z@}%
  {-1.25ex\@plus -1ex \@minus -.2ex}%
  {1.5ex \@plus .2ex}%
  {\normalfont\slshape}}
\renewcommand{\subsection}{\@startsection{subsection}{2}{\z@}%
  {-3.25ex\@plus -1ex \@minus -.2ex}%
  {1.5ex \@plus .2ex}%
  {\normalfont\bfseries\slshape}}
\renewcommand{\section}{\@startsection{section}{1}{\z@}%
  {-5.25ex\@plus -1ex \@minus -.2ex}%
  {1.5ex \@plus .2ex}%
  {\normalfont\bfseries}}
\makeatother
%
\renewcommand\thesection{\arabic{section}.}
\renewcommand\thesubsection{\thesection\arabic{subsection}}
\renewcommand\thesubsubsection{\thesubsection\arabic{subsubsection}}
%
\renewcommand{\headrulewidth}{0pt}

\usepackage{listings}

\newenvironment{mylisting}
{\begin{list}{}{\setlength{\leftmargin}{1em}}\item\scriptsize\bfseries}
{\end{list}}

\newenvironment{mytinylisting}
{\begin{list}{}{\setlength{\leftmargin}{1em}}\item\tiny\bfseries}
{\end{list}}

\usepackage{listings}

\definecolor{darkblue}{rgb}{0,0,0.5}
\definecolor{shadecolor}{rgb}{1,1,0.95}
\definecolor{shade}{rgb}{1,1,0.95}


\lstset{ %
language=R,   % the language of the code
basicstyle=\footnotesize, % the size of the fonts that are used for the code
numbers=left,   % where to put the line-numbers
numberstyle=\footnotesize, % the size of the fonts that are used for the line-numbers
stepnumber=100,   % the step between two line-numbers. If it's 1, each line 
    % will be numbered
numbersep=5pt,   % how far the line-numbers are from the code
backgroundcolor=\color{shade}, % choose the background color. You must add \usepackage{color}
showspaces=false,  % show spaces adding particular underscores
showstringspaces=false,  % underline spaces within strings
showtabs=false,   % show tabs within strings adding particular underscores
frame=single,   % adds a frame around the code
tabsize=2,   % sets default tabsize to 2 spaces
captionpos=b,   % sets the caption-position to bottom
breaklines=true,  % sets automatic line breaking
breakatwhitespace=false, % sets if automatic breaks should only happen at whitespace
title=\lstname,   % show the filename of files included with \lstinputlisting;
    % also try caption instead of title
escapeinside={\%*}{*)},  % if you want to add a comment within your code
morekeywords={*,...}  % if you want to add more keywords to the set
}

%
\title{Sargasso Sea Evaluation of Ecosystem Indicators: Data Poor Stocks.}
%
\author{Laurence T. Kell\footnote{Centre for Environmental Policy, Imperial College London, London SW7 1NE, UK.} \space and
        Brian E. Luckhurst\footnote{2-4 Via della Chiesa, 05023 Acqualoreto (TR), Umbria, Italy.} \space}
%
\date{}
%

\begin{document}
\maketitle

\onehalfspacing
\lhead{\normalsize\textsf{SCRS/20/xxx}}
\rhead{}

\maketitle
% gets headers on title page ...
\thispagestyle{fancy}
% ... but not on others
\pagestyle{empty}

%
\begin{abstract}

\textit{}

\end{abstract}

\Keywords{}

\section{Introduction}

\section{Material and Methods}

\subsection{ICES Assessment Categories}

\begin{description}
 \item[Category 1: stocks with quantitative assessments]
Full analytical assessments and forecasts as well as stocks with quantitative assessments based on production models.

 \item[Category  2: stocks  with  analytical  assessments  and  forecasts  that  are  only  treated  qualitatively]
Quantitative assessments and forecasts which for a variety of reasons are considered indicative of trends in fishing mortality, recruitment, and biomass.
 \item[Category 3: stocks for which survey based assessments indicate trends] 
Survey or other indices are available that provide reliable indications of trends in stock metrics, such as total mortality, recruitment, and biomass.

 \item[Category 4: stocks for which only reliable catch data are available] 
Time series of catch can be used to approximate MSY.

 \item[Category 5: landings only stocks] 
Only landings data are available.

 \item[Category 6: negligible landings stocks and stocks caught in minor amounts as bycatch] Landings are negligible in comparison to discards and stocks that are  primarily caught as bycatch species in other targeted fisheries 
 
\end{description}

\subsection{Data Poor Methods}

Indicators and reference point may be biased and have poor precision due to uncertainty about life history parameters. Bias and precision are both important factors to consider when assessing fish stocks. Bias reflects how close an estimate is to a known value; precision reflects reproducibility of the estimate. For example, if an assessment is to be re-conducted every year to monitor the impact of a management measure, a precise but biased method would be able to detect a trend better than an unbiased but imprecise method. Like a scientific instruments, this trade-off require calibration to correct for the bias, and such calibration can be explored using closed-loop simulations such as management strategy evaluation, where the choice of parameters and reference points in a management procedure are tuned (i.e. calibrated) to meet the desired management objectives as represented by the operating model. T

hus, a biased method may be preferable to one that is less biased, but more imprecise. Alternatively, imprecision can be addressed through the choice of the percentile (e.g., median being the 50\% percentile value) for the derived model output used by management (e.g., catch or SPR); assuming that the true value is contained within the parameter distribution. For example, instead of taking the median value, one could instead use the derived model output associated with the 40th percentile to incorporate risk tolerance as reflected in the calculated imprecision. Such an approach (Ralston et al. 2011) is used in fisheries management systems to directly incorporate scientific uncertainty (both bias and imprecision), and can also be explored and tuned using MSE. 

Assessing stocks using only catch and life-history data started many years ago with the development of Stock Reduction Analysis (SRA; Kimura and Tagart 1982; Kimura et al. 1984)Kimura and Tagart, 1982; Kimura et al., 1984). Since then, SRA has been extended to estimate productivity and reconstruct historical abundance trends by making assumptions about final biomass relative to unfished or initial biomass (i.e., stock depletion; Thorson and Cope 2015). SRA has been further extended to incorporate stochastic variability in population dynamics (Stochastic-SRA; Walters et al. 2006), a flexible shape for the production function (Depletion-Based SRA; Dick and MacCall 2011), prior information regarding resilience and population abundance at the start of the catch time series (Catch-Maximum Sustainable Yield, Catch-MSY; Martell and Froese 2013), bayesian approches (CMSY, Froese et al. 2017), and age-structured population dynamics (Simple Stock Synthesis, SSS; Cope 2013). Despite these differences, this family of catch-only models share a common dependence on prior assumptions about final stock depletion. Simulation testing has previously indicated that these methods perform well only when assumptions regarding final relative abundance are met (Wetzel and Punt 2015). Unsurprisingly, because final stock depletion is a prior assumption, the methods perform differently under different stock depletion levels (i.e., highly depleted or slightly depleted stocks, Walters et al. 2006) or under different harvest history or catch trends.

Pons et al., 2020 simulation tested a variety of methods, both length and catch based, ...


\section{Results}

~

Figure \ref{fig:1}

Figure \ref{fig:2}

Figure \ref{fig:3}

Figure \ref{fig:4}

Figure \ref{fig:5}

Figure \ref{fig:6}


\newpage
\bibliography{/home/laurence/Desktop/refs.bib} 
\bibliographystyle{abbrvnat} 

\end{document}

    
